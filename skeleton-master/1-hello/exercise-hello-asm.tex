\documentclass{scrartcl}

\usepackage{url}
\usepackage[hidelinks]{hyperref}
\usepackage{listings}

\title{Hello World in Assembly}
\author{Emmanuel Benoist, Christian Grothoff, Pascal Mainini}
\date{Fall Term 2019/2020}
\begin{document}

\maketitle

This exercise must be done inside the clone of your own fork of the skeleton git repository.

\section{Presentation of the files}



The file \texttt{hello.asm} contains an assembler file that needs to be completed. This is the main file you need to change.

The file \texttt{makefile} contains the instructions for the compilation. 

\section{Goal of the exercise}

The goal of this exercise is to work on a small Assembler file, to compile and run it. It should write the string ``Hello world!'' with a Line Feed in the end on the standard output.

\section{Expected Output}

The program must have the following behaviour:

\begin{lstlisting}
machine-:$ ./hello
Hello world! 
machine-:$ ./hello
Hello world!
\end{lstlisting}

\section{Upload the result to the Git repository}
You must upload the file you wrote \texttt{hello.asm} to the repository.

\begin{lstlisting}
machine-:$ git add hello.asm
....
machine-:$ git commit -m"solution for exercise 1"
....
machine-:$ git push origin master
\end{lstlisting}%$

\section{Get feedback}
You can inspect the results of the automated tests for your program inside gitlab CI (\url{https://gitlab.ti.bfh.ch}, go to \texttt{CI/CD > Jobs} in your forked project). There, you will also find the amount of points given.

\end{document}
