\documentclass{scrartcl}

\usepackage{url}
\usepackage[hidelinks]{hyperref}
\usepackage{listings}

\title{Sort a List of Strings in C}
\author{Emmanuel Benoist, Christian Grothoff, Pascal Mainini}
\date{Fall Term 2019/2020}
\begin{document}

\maketitle

This exercise must be done inside the clone of your own fork of the skeleton git repository.

\section{Presentation of the files}



The file \texttt{sort.c} contains a C file that must be completed. This is the main file you have to change.

The file \texttt{makefile} contains the instructions for the compilation. 

\section{Goal of the exercise}

The goal of this exercise is to write a program in C which sorts the lines of its input. By default, lines are sorted in ascending order. The program must support an optional \texttt{-r} parameter which causes sorting to be done in descending order. For example, ascending mode looks like this:

\begin{lstlisting}
a
c
b
\end{lstlisting}

The output must be
\begin{lstlisting}
a
b
c
\end{lstlisting}


\section{Expected Output}

The program must sort lines in ascending or descending order. For testing, enter lines interactively and terminate input with \texttt{<Control-D>}.

\subsection{Output without optional parameter}

\begin{lstlisting}
machine-:$ ./sort
a
v
b
<Control-D>
a
b
v

machine-:$ ./sort
bonjour
hallo
gruesseuch
<Control-D>
bonjour
gruesseuch
hallo
\end{lstlisting}

\subsection{Output with optional parameter}

\begin{lstlisting}
machine-:$ ./sort -r
a
v
b
<Control-D>
v
b
a

machine-:$ ./sort -r
bonjour
hallo
gruesseuch
<Control-D>
hallo
gruesseuch
bonjour
\end{lstlisting}

\section{Upload the result to the Git repository}
You must upload the file you wrote, \texttt{sort.c}, to the repository.

\begin{lstlisting}
machine-:$ git add sort.c
....
machine-:$ git commit -m"solution for exercise 4"
....
machine-:$ git push origin master
\end{lstlisting}%$

\section{Get feedback}
You can inspect the results of the automated tests for your program inside gitlab CI (\url{https://gitlab.ti.bfh.ch}, go to \texttt{CI/CD > Jobs} in your forked project). There, you will also find the amount of points given.

\end{document}
