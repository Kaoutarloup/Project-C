\documentclass{scrartcl}

\usepackage{url}
\usepackage[hidelinks]{hyperref}
\usepackage{listings}

\title{Implement a Small UNIX Shell}
\author{Emmanuel Benoist, Christian Grothoff, Pascal Mainini}
\date{Fall Term 2019/2020}
\begin{document}

\maketitle

This exercise must be done inside the clone of your own fork of the skeleton git repository.

\section{Presentation of the files}



The file \texttt{shell.c} contains a C file that must be completed. This is the main file you have to change.

The file \texttt{makefile} contains the instructions for the compilation.

\section{Goal of the exercise}

The goal of this exercise is to write a program in C, which emulates a Unix shell (a small one).

Your shell must:
\begin{enumerate}
\item  Print a shell prompt (\texttt{"\$ "}) for each command.

Example:

\begin{lstlisting}
$
\end{lstlisting}


\item  Read a stream of simple commands to execute from stdin.

Example:

\begin{lstlisting}
$ echo Hello
Hello
$ expr 1 + 3
4
\end{lstlisting}


\item  Support setting shell variables and expansion.

Example:

\begin{lstlisting}
$ A=5
$ echo $A
5
$ echo ${A}BC
5BC
$ A=$A$A
echo $A
55
\end{lstlisting}%$



\item  Support checking the status of the last command via \$?.

Example:
\begin{lstlisting}
$ true
$ echo $?
0
$ false
$ echo $?
1
\end{lstlisting}



\item  Support launching processes in the background with '\&'.

  Example:

  \begin{lstlisting}
$ firefox &
$
\end{lstlisting}


(here, Firefox should run and your shell should accept additional
commands as inputs while Firefox is running).  Without '\&', the
shell must only accept new commands after the previous command
has terminated!


\item Support redirection of stdin, stdout and stderr with \texttt{">"}, \texttt{"<"} and \texttt{"2>"}.

  Example:


  \begin{lstlisting}
$ echo Hello > test.txt
$ cat - < test.txt
Hello
\end{lstlisting}

\item  Support the pipeline operator \texttt{"|"}.

  Example:

  \begin{lstlisting}
$ echo Hello | tr l L | tr o ?
HeLL?
\end{lstlisting}%$
\end{enumerate}

A passing grade will be awarded for completing 6/7 of the features.


\section{Upload the result to the Git repository}
You must upload the file you wrote \texttt{shell.c} to the repository.

\begin{lstlisting}
machine-:$ git add shell.c
....
machine-:$ git commit -m"solution for exercise 5"
....
machine-:$ git push origin master
\end{lstlisting}%$

\section{Get feedback}
You can inspect the results of the automated tests for your program inside gitlab CI (\url{https://gitlab.ti.bfh.ch}, go to \texttt{CI/CD > Jobs} in your forked project). There, you will also find the amount of points given.

\end{document}
